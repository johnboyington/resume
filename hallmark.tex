%%%%%%%%%%%%%%%%%%%%%%%%%%%%%%%%%%%%%%%%%
% Medium Length Graduate Curriculum Vitae
% LaTeX Template
% Version 1.1 (9/12/12)
%
% This template has been downloaded from:
% http://www.LaTeXTemplates.com
%
% Original author:
% Rensselaer Polytechnic Institute (http://www.rpi.edu/dept/arc/training/latex/resumes/)
%
% Important note:
% This template requires the res.cls file to be in the same directory as the
% .tex file. The res.cls file provides the resume style used for structuring the
% document.
%
%%%%%%%%%%%%%%%%%%%%%%%%%%%%%%%%%%%%%%%%%
\vskip-20pt%
%----------------------------------------------------------------------------------------
%	PACKAGES AND OTHER DOCUMENT CONFIGURATIONS
%----------------------------------------------------------------------------------------

\documentclass[margin, 10pt]{res} % Use the res.cls style, the font size can be changed to 11pt or 12pt here

\usepackage{helvet} % Default font is the helvetica postscript font
%\usepackage{newcent} % To change the default font to the new century 
%schoolbook %postscript font uncomment this line and comment the one above

\setlength{\textwidth}{5.1in} % Text width of the document

\begin{document}

%----------------------------------------------------------------------------------------
%	NAME AND ADDRESS SECTION
%----------------------------------------------------------------------------------------

\moveleft.5\hoffset\centerline{\Large\bf John Charles Boyington} % Your name at the top
 
\moveleft\hoffset\vbox{\hrule width\resumewidth height 1pt} \vskip-10pt%
\moveleft\hoffset\vbox{\hrule width\resumewidth height 
0.15pt}\smallskip 
%Horizontal line after name; adjust line thickness by changing the '1pt'
 
\moveleft.5\hoffset\centerline{777 Hoopes Ave., Apt. J104 \hfill (785) 533-4260} % Your address
\moveleft.5\hoffset\centerline{Idaho Falls, ID 83401 \hfill mjcb@ksu.edu}

%----------------------------------------------------------------------------------------

\begin{resume}
 
\section{OBJECTIVE}  

% YO, REVISIT
Gain expertise in the use of production level data analytic software, data visualization techniques, and business strategies by leveraging my academic background in scientific computing, artificial intelligence, and software development.

%----------------------------------------------------------------------------------------
\section{EDUCATION}

{\sl Master of Science,} Nuclear Engineering \hfill Expected August 2019\\
Kansas State University, Manhattan, KS \\
Cumulative GPA: 3.625/4.000

{\sl Bachelor of Science,} Mechanical Engineering \hfill December 2016 \\
Japanese Language Minor \\
Kansas State University, Manhattan, KS \\
Cumulative GPA: 2.993/4.000


%----------------------------------------------------------------------------------------
\section{RELEVANT \\ EXPERIENCE}


% INL
{\sl Nuclear Plant Safety Systems Intern} \hfill Spring 2019 - Present \\
Idaho National Laboratory, Idaho Falls, ID
\begin{itemize}
    \item Currently developing software for modernization of neutron spectrometric measurement analysis and calculation of dosimetric quantities using object-orientated design.
\end{itemize}

% GRA
{\sl Graduate Research Assistant} \hfill Spring 2017 - Spring 2019 \\
Mechanical and Nuclear Engineering, Kansas State University, Manhattan, KS
\begin{itemize}
    % ai class
    \item Applied advanced data processing techniques and TensorFlow artificial neural networks for the creation of several game-playing and image-classifying agents, including agent implemented in PySC2 StarCraft II Learning Environment.
    % wrote efor
    \item Built and applied genetic-algorithm-based framework for the optimization of a neutron beam filter.
    % visualizing filter
    \item Crafted data visualization scheme for multi-dimensional neutron beam filter dataset that allowed trends to be observed and meaningful conclusions to be drawn.
    % the hot fuel element
    \item Applied 3D data-visualization to inform experimental team on potential radiological hazards of TRIGA fuel element removal and analysis experiment.
    % technical writing
    \item Drafted several technical documents detailing experiment and software design, implementation, and analysis, including drafts for United States Department of Energy.
    % need some stats
    \item Applied statistical methods and regression techniques to the analysis of reactor experiment results.
    % made model from scratch full triga model
    %\item Contributed to design and implementation of reactor simulation API and subsequent development of KSU Triga Mark II nuclear reactor model.
    % this used object oriented framework
    %\item Scripted post-processing of multiple in-core and ex-core, neutron spectrometric measurement experiment analyses using carefully-designed, object-oriented framework.
    % made several apis
    %\item Constructed Python-based wrappers for multiple codes used in neutron transport simulation, spectral deconvolution, and data visualization.
\end{itemize}

% GTA
{\sl Graduate Teaching Assistant} \hfill Fall 2017 - Spring 2018 \\
Mechanical and Nuclear Engineering, Kansas State University, Manhattan, KS \\
\begin{itemize} \itemsep -2pt % Reduce space between items
    \item Led laboratory and help sessions for undergraduate and graduate scientific computing programming courses.
    \item Developed course materials (quizzes, laboratories, homework and exams) for students.
\end{itemize}


%----------------------------------------------------------------------------------------
%\section{COMPUTER \\ SKILLS} 

%{\sl Languages:}
%Python, \LaTeX, Bash, C++, Fortran, Matlab, VBA

%{\sl Software:}
%TensorFlow, Visit, Git, MCNP6, ADVANTG, UMG3.3, SandIV, NJOY, Genie 2000, Microsoft Office Suite

%{\sl Operating Systems:}
%Linux Mint, Windows 10, Windows 7

%----------------------------------------------------------------------------------------
%\section{PUBLISHED \\ WORKS}
%Conference Transactions
%\begin{itemize}
%\item John C. Boyington, Richard L. Reed, Ryan Ullrich, Jeremy A. Roberts. (2017, Nov). ``Gamma-Ray and Thermal-Neutron Filter Design for a TRIGA Penetrating Beam Port", 2017 Winter Meeting of American Nuclear Society, Washington D.C.
%\end{itemize}


\end{resume}
\end{document}
